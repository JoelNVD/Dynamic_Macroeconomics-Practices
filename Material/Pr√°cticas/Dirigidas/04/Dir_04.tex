\documentclass[11pt,a4paper]{article}


\usepackage[latin1]{inputenc}
\usepackage[T1]{fontenc}
\usepackage[spanish]{babel}

\usepackage{amsfonts}
\usepackage{amsmath}
\usepackage{amssymb}

\usepackage{graphicx}
\usepackage[left=2cm,right=2cm,top=2cm,bottom=2cm]{geometry}
\usepackage{fancyhdr}
\usepackage{parskip}
\usepackage{color}
\usepackage{multicol}
\usepackage[shortlabels]{enumitem} %Para comnfigurar la numeraci�n de en enumerate


\textwidth=450pt \textheight=620pt \oddsidemargin=0in
\topmargin=-10pt
\pagestyle{fancy} \rhead{\scriptsize{\textbf{C�digo del Curso:} EP-3072} \\
	\textbf{Fecha:} 11/08/2021 \& 2021-I \hspace{0.03cm}}
\lhead{\scriptsize{\textbf{Profesor:} Joel Vicencio-Damian}  \\
	\textbf{Curso:} Macroeconom�a Din�mica-Pr�cticas}
\newcommand{\re}[1]{\smallskip\textsf{\textbf{Respuesta}} \begin{sf}\\ #1 \end{sf} \bigskip}
\newcommand{\ay}[1]{ \scriptsize{\textsl{Hint: #1}}\normalsize{}}
\newcommand{\pr}[2]{\frac{\partial #1}{\partial #2}}

\begin{document}
%PRESENTACI�N--------------------------------------------------------------------------------	
	\begin{center}
		{\Large {\textbf{Practica Dirigida N$^{\circ}$4}}}
		
		\small{Semana 6}
		
		\textsc{C�lculo de Variaciones II}
		
		Condici�n de Transversalidad \& Diagrama de Fases
		
	\end{center}
	
%EJERCICIOS---------------------------------------------------------------------------------
	\begin{enumerate}
		\item Condiciones de Transversalidad
				\begin{enumerate}
					\item [1a)] $V(y)=\int\limits_{0}^{2}(t^2+\dot{y}^2)dt,\, \textup{con}\enskip y(0)=4,\, y(2)=y_t\enskip (y_t\, es\, libre)$
					\item [1b)] $V(y)=\int\limits_{0}^{T}(t+\dot{y}^2)dt,\, \textup{con}\enskip y(0)=4,\, y(T)=5\enskip \textup{y}\enskip T\, es\, libre$
					\item [1c)] $V(y)=\int\limits_{0}^{T}(t\dot{y}+\dot{y}^2)dt,\, \textup{con}\enskip y(0)=1,\, y(T)=10\enskip \textup{y}\enskip T\, es\, libre$
					\item [1d)] $V(y)=-\int\limits_{0}^{T}(1+\dot{y}^2)^{0.5}dt,\, \textup{con}\enskip y(0)=1,\, y(T)=2-3T$
					\item [1e)] $V(y)=\int\limits_{0}^{T}(1+\dot{y}^2)^{0.5}dt,\, \textup{con}\enskip y(0)=1,\, y(T)=2-T$
					\item [1f)]$V(y) = \int \limits_{0}^{\infty} e^{-\rho t}(y^{2}+ay+b\dot{y}+c\dot{y}^{2})dt, \enskip y(0)=d, \enskip (a, b, c, d, \rho > 0)$
				\end{enumerate}
		\item Diagrama de Fases
				\begin{enumerate}[2a)]
					\item Obtengan, matem�ticamente, el diagrama de fases de la siguiente expresi�n
						\begin{align*}
							\dot{x} & = -x + 2y \\
							\dot{y} & = -3y
						\end{align*}
					\item Se define el siguiente sistema de ecuaciones diferenciales lineales
						\begin{align*}
							\dot{y} & = ay+bx+h\\
							\dot{x} & = cy+dx+k
						\end{align*}
					  Dibujar el diagrama de fases seg�n las siguientes condiciones:
						\begin{align*}
							 \textup{Caso I: } & a>0,\enskip b<0,\enskip c>0,\enskip d>0\\ 
							 \textup{Caso II: } & a<0,\enskip b>0,\enskip c>0,\enskip d<0\\
							 \textup{Caso III: } & a>0,\enskip b<0,\enskip c<0,\enskip d<0\\
							 \textup{Caso IV: } & a=0,\enskip b<0,\enskip c>0,\enskip d=0
						\end{align*}
					\item Realice el diagrama de fases de los siguientes sistemas de sistemas de ecuaciones diferenciale.
						\begin{enumerate}[2c1)]
							\item $\dot{y}=-3y+y^2+2$
							\item $\dot{x}=3x-18 	 \hspace{1.1cm} \dot{y}=-2y+16$
							\item $\dot{x}=y-x^{2}+3 \qquad \dot{y}=y-x+1$
							\item $\dot{x}=y-x^{3} 	 \hspace{1.3cm} \dot{y}=1-xy$
						\end{enumerate}
				\end{enumerate}
			\item \textbf{Aplicaciones econ�micas}
				\begin{enumerate}[3a)]
					\item \textbf{Modelo IS-LM}\\
							$$\dot{y}=a[E(Y-T,r)+G-Y]=f(Y,r)$$
							$$\dot{r}=b\left[L(Y,r)-\frac{M}{P}\right]=g(Y,r)$$
						Donde $Y$ es el nivel de producci�n, $r$ es la tasa de inter�s, $E$ es igual a la suma de los gastos en consumo e inversi�n, $G$ es el gasto p�blico, $T$ son los pagos por impuestos y $P$ el nivel de precios. Las constantes $a$ y $b$ son positivas y representan la velocidad de ajuste del mercado de bienes y del mercado de dinero, respectivamente.\
					
						Se asume $G,\enskip T,\enskip M$ y $P$ como fijos, adem�s las funciones de gastos $E$ y la demanda por dinero cumplen las siguientes propiedades.
							$$0<E_y<1, \quad E_r<0, \quad L_y>0, \quad L_r<0$$
						Dibujar el diagrama de fases para este modelo y establecer los casos extremos.
					\item \textbf{Interacci�n Demanda-Oferta}\\
						Supongamos que la demanda para un bien depende de su precio $p$ y la oferta de su precio esperado $p^e$. Las cantidades demandadas y ofertadas son
							$D(p)$ y $O(p^e)$ donde $D$ y $O$ son funciones tal que $D'(p)<0$ y $O'(p^e)>0$.
						Supongamos que el precio $p$ reacciona al desequilibrio del mercado, con su tasa de cambio proporcional a su desequilibrio. Esto es
							$$\dot{p}=\alpha [D(p)-O(p^e)], \quad \alpha>0 \enskip \textup{constant}$$
						Asumimos que el precio esperado tiene una tasa de cambio proporcional a su adaptaci�n en el mercado.
							$$\dot{p^e}=\beta (p-p^e), \quad \beta>0 \enskip \textup{constante}$$
					\item \textbf{Explotaci�n �ptima de peces}\\
						Suponga que una poblaci�n de ``$N(t)$"peces en cierto lago, crece a la siguiente tasa:$$\dot{N}(t)=aN(t)-bN^2(t)$$
						En ausencia de actividad de extracci�n. En una comunidad cercana al lago se consume una cantidad ``$C(t)$" de pescado, que brinda una utilidad igual ``$U(c)$" ($U'(c)>0, \enskip U''(c)<0$) y altera el crecimiento de la biomasa de la siguiente forma:$$\dot{N}(t)=aN(t)-bN^2(t)-C(t)\qquad (a,b>0)$$
						El objetivo de la comunidad es maximizar las utilidades futuras descontadas con la tasa $\rho$: 	$$\dot{V}(c)=\int\limits_{0}^{\infty}e^{-\rho t}U(c)dt$$
						Considerando la poblaci�n actuaal de pesces $N_0=\frac{a}{b}$, se le pide resolver el siguiente problema de c�lculo de variaciones mediante el diagrama de fases.
				\end{enumerate}
			\item Problema integrador\\
					\textbf{Aplicaci�n:\textit{ Modelo de crecimiento de Ramsey-Cass-Koopmans}}
						\begin{align*}
							\textup{Max}\qquad V[y]&=\int\limits_{0}^{\infty}U(c)e^{-\rho t}dt\\
							\textup{Sujeto a}\qquad c&=Ak-\dot{k}-\delta k \\
							k(0)&=10 \\
							A-\delta&>0	 \\
							A-\delta-\rho&<0
						\end{align*}
					Donde $U(c)=\ln c$ y la funci�n de producci�n $f(k)=Ak$\
					
					Resuelva el problema incluyendo:
						\begin{enumerate}[4a)]
							\item Condiciones Necesarias
							\item Condiciones de Transversalidad
							\item Condiciones Suficientes
							\item Diagrama de Fases
						\end{enumerate}
	\end{enumerate}
\end{document}