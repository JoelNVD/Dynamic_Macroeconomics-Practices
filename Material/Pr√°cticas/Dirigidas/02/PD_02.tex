\documentclass[11pt,a4paper]{article}


\usepackage[utf8]{inputenc}
\usepackage[T1]{fontenc}
\usepackage[spanish]{babel}

\usepackage{amsfonts}
\usepackage{amsmath}
\usepackage{amssymb}

\usepackage{graphicx}
\usepackage[left=2cm,right=2cm,top=2cm,bottom=2cm]{geometry}
\usepackage{fancyhdr}
\usepackage{parskip}
\usepackage{color}
\usepackage{multicol}
\usepackage[shortlabels]{enumitem} %Para comnfigurar la numeración de en enumerate


\textwidth=450pt \textheight=620pt \oddsidemargin=0in
\topmargin=-10pt
\pagestyle{fancy} \rhead{\scriptsize{\textbf{Código del Curso:} EP-3072} \\
	\textbf{Fecha:} 14/07/2021 \& 2021-I \hspace{0.03cm}}
\lhead{\scriptsize{\textbf{Profesor:} Joel Vicencio-Damian}  \\
	\textbf{Curso:} Macroeconomía Dinámica}
\newcommand{\re}[1]{\smallskip\textsf{\textbf{Respuesta}} \begin{sf}\\ #1 \end{sf} \bigskip}
\newcommand{\ay}[1]{ \scriptsize{\textsl{Hint: #1}}\normalsize{}}
\newcommand{\pr}[2]{\frac{\partial #1}{\partial #2}}

\begin{document}
% PRESENTACIóN--------------------------------------------------------------------------------	
	\begin{center}
		{\Large {\textbf{Practica Dirigida N$^{\circ}$2}}}
		
		\small{Semana 2}
		
		\textsc{Ecuaciones En Diferencias}
		
		Ecuaciones En Diferencias de primer y segundo orden
		
	\end{center}

% EJERCICIOS---------------------------------------------------------------------------------
\begin{enumerate}
		\item Demostrar la solución general de una ecuación en diferencia y simular valores en Python de la siguiente expresión: $$x_{t} = ax_{t-1} + b$$
			\begin{equation*}
				\begin{matrix}
					\textup{Monótona y Divergente}: 	 & a = 7 &, 		   & b=16 &, &x_{0}=5\\
					\\
					\textup{Monótona y Convergente}:	 & a = \frac{1}{3} &,  & b=6  &,  &x_{0}=1\\
					\\
					\textup{Oscilante y Divergente}:  & a = -2 &, 		   & b=1  &,  &x_{0}=1\\
					\\
					\textup{Oscilante y Convergente}: & a = -\frac{1}{4} &, & b=5  &,  &x_{0}=2
				\end{matrix}
			\end{equation*}
		\item Diagrama de fases para Ecuaciones En Diferencia\\
		Resolver las siguientes ecuaciones en diferencia de primer orden, describiendo el procedimiento y simular los resultados en Python
			\begin{multicols}{2}
				\begin{enumerate}[a)]
					\item $y_{t} = \frac{y_{t-1}}{2}+5$
					\item $y_{t} = 5y_{t-1}$
					\item $y_{t} = y_{t-1}^{0.5}$ %Dowling 397 converge
					\item $y_{t} = y_{t-1}^{3}$ %Dowling 405 diverge
					\item $y_{t} = y_{t-1}^{-0.25}$ %Dowling 406 converge
					\item $y_{t} = y_{t-1}^{-1.5}$ %Dowling 407 diverge
				\end{enumerate}
			\end{multicols}
			\textbf{Reto: Ver el esquema de telaraña con la pregunta 1}
		\item Resolver las siguientes ecuaciones en diferencias de segundo orden
				\begin{enumerate}[a)]
					\item $y_{t+2}-11_{t+1}+10y_{t}=27 \quad y_{(0)}=2 \quad y_{(1)} = 53$ %Dowling 421
					\item $y_{t}-10_{t-1}+25y_{t-2}=8 \quad y_{(0)}=1 \quad y_{(1)} = 5$ %Dowling 421
				\end{enumerate}
		\item \textbf{Aplicaciones Económica}
			\begin{enumerate}
				\item \textbf{Modelo de determinación de ingresos retesados}
					$$C_t=90+0.8Y_{t-1} \qquad T_t=50 \qquad Y_0 = 1200$$
				\item \textbf{Modelo de la telaraña}
					$$Q_{dt}=180-0.75P_t \qquad Q_{st}=-30+0.3P_{t-1 \qquad P_0=200}$$
				\item \textbf{Modelo de crecimiento de Harrod}
					$$I_t=2.66(Y_t-Y_{t-1}) \qquad S_t = 0.16Y_t \qquad Y_0=9000$$
			\end{enumerate}
\end{enumerate}
\end{document}